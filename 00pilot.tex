% rubber: set program xelatex

\documentclass{screenplay}
\usepackage{xltxtra,fontspec,xunicode}
\setmainfont{NanumGothic}
\setmonofont{UnGraphic}

\title{프로그래머들 (The Programmers) -- Pilot}

\author{이인호}

\address{
chajath@gmail.com
}

\begin{document}
    \coverpage
    
    \section{Act 1}
    
    \fadein
    \intslug[DAY]{회사 개발 사무실}
    개발 사무실에서 김석호와 최진수가 작업에 몰두하고 있다. 
    \begin{dialogue}[짜증 내는 말투로]{김석호}
        아침부터 되는 일이 없네.
    \end{dialogue}
    \begin{dialogue}{최진수}
        왜 뭐가 잘 안되?
    \end{dialogue}
    \begin{dialogue}{김석호}
        어. LaTeX 컴파일 에러가 나는데 어디가 잘못됐는지 아무리 봐도 모르겠네. 패키지가 다 설치되 있는 것 같은데 자꾸 에러가 나. 
    \end{dialogue}
    최진수 한숨을 쉬며 자리에서 일어나서 김석호 자리로 온다.
    \begin{dialogue}[어이없다는 듯이]{최진수}
        그러니까 왜 쓸데없이 LaTeX를 써? 그건 대학교 연구실에서 쓰는 거고 여긴 회사야. 이렇게 LaTeX으로 문서화를 하면 후임자들이 이걸 관리해야 되잖아. 너 개인 취향만 따져서 이런 걸 쓰면 어떻게. 
    \end{dialogue}
    \begin{dialogue}[반색하며]{김석호}
        난 프로그래머야. 회사가 정신이 돌지 않는 이상 내 후임자도 프로그래머일 것이고, 그런데 뭐가 문제야. LaTeX 못쓰는 프로그래머도 있어?
        \paren{잠시 머뭇거리다}
        근데 넌 LaTeX쓰는거 한번도 못 봤다? 너 LaTeX쓸줄 알지?
    \end{dialogue}
    \begin{dialogue}{최진수}
        야 내가 언제 못쓴다고 했냐 그냥 회사에서는 잘 안 쓰는 툴이라 이거지.
        \paren{헛기침}
        어쨌든, 근데 뭘 만들려고 LaTeX을 써?
    \end{dialogue}
    \begin{dialogue}[웃으며]{김석호}
        일기! 이제 매일 매일 일기를 쓸려고.
    \end{dialogue}
    \begin{dialogue}[황당하다는 듯이]{최진수}
        참네. 일기를 쓰는데 LaTeX이 왜 필요해?
    \end{dialogue}
    \begin{dialogue}[순진하게]{김석호}
        일기에 당연히 LaTeX을 써야지! 안 그러면 선형대수학 공식이나 항렬은 어떻게 집어넣을껀데?
    \end{dialogue}
    \begin{dialogue}{최진수}
        아니 일기장에 도대체 그런게 왜 들어가? 넌 도대체 무슨 생각을 하면서 사는 거야?
    \end{dialogue}
    \begin{dialogue}{김석호}
        \paren{잠시 최진수의 시선을 의식하며}
        그런 한심하다는 듯한 얼굴 표정 짓지 마. 내가 일기장에 뭘 쓰던지, 그건 내 자유잖아? 그런 걸로 사람을 판단하지 마!
    \end{dialogue}
    \begin{dialogue}{최진수}
        너무 민감하게 생각하지는 마. 하긴 너처럼 현실과 동떨어진 세계관을 가지고 살아가려면 그런 것도 필요한지도 모르겠네.
    \end{dialogue}
    \begin{dialogue}{김석호}
        아니 내가 현실과 동떨어진 세계관을 가졌다니 무슨 말을 하는 거야?
    \end{dialogue}
    두 사람이 대화를 하는 사이 주위 직원들이 하나 둘 모여든다.
    \begin{dialogue}{최진수}
        세상은 니가 생각하는 그런 곳이 아니야. 지금 제일 많이 쓰이는 운영체제가 뭐야?
    \end{dialogue}
    \begin{dialogue}{김석호}
        우분투!
    \end{dialogue}
    \begin{dialogue}{최진수}
        땡! 윈도우즈!
        \paren{김석호 좌절하는 표정}
        제일 많이 쓰는 웹 브라우저는?
    \end{dialogue}
    \begin{dialogue}{김석호}
        파이어폭스!
    \end{dialogue}
    \begin{dialogue}{최진수}
        틀렸어! 인터넷 익스플로러야! 
        \paren{사악한 미소를 지으며}
        너 혹시 인터넷 뱅킹 해봤어?
    \end{dialogue}
    주위에 모인 직원들이 최진수를 말린다
    \begin{dialogue}{직원1}
        그만해 최진수. 왜 일을 키우는 거야?
    \end{dialogue}
    \begin{dialogue}{직원2}
        동료한테 왜 이렇게 잔인하게 구는 거야?
    \end{dialogue}
    \begin{dialogue}{김석호}
        당연히 해봤지. 그건 왜?
    \end{dialogue}
    \begin{dialogue}{최진수}
        그때 넌 무슨 브라우저를 쓰고 있었지?
    \end{dialogue}
    \begin{dialogue}{김석호}
        음... 브라우저? 뭘 쓰고 있었더라.
    \end{dialogue}
    \begin{dialogue}{최진수}
        당연히 기억 못하겠지. 왜 그런지 알아?
        \paren{김석호 고개를 젓는다}
        그건 그 기억이 너무 트라우마틱해서 무의식적으로 기억을 짓누르고 있기 때문이야. 
        \paren{손가락으로 머리를 가리키며}
        너의 방어기제가 작동하고 있는거지.
    \end{dialogue}
    \begin{dialogue}[서서히 고통에 잠기면서 얼굴을 손으로 감싼다]{김석호}
        무슨.. 무슨 소리를 하고있는 거야 지금?
    \end{dialogue}
    \begin{dialogue}{최진수}
        너는 인터넷 뱅킹을 할려고 인터넷 익스플로러를 썼었어.
    \end{dialogue}
    \begin{dialogue}{김석호}
        그럴리가... 그럴리가 없어. 네 이놈 그 입 다물지 못할까!
    \end{dialogue}
    \begin{dialogue}{최진수}
        이거 봐 말투부터 이상해지고 있잖아. 그게 왜 그런지 알아?
        \paren{다가가서 속삭이는 말투로}
        그건 니가 미쳐가고 있기 때문이야
    \end{dialogue}
    주위에 직원들 웅성거리는 소리. 몇몇 직원은 최진수의 팔을 붙들며 말리는 시늉을 한다.
    \begin{dialogue}{김석호}
        하하하하. 내가 미쳤다고. 내가 미쳤다고!
        \paren{팔을 휘두르며}
        미친 건 내가 아니라 너야! 내가 미쳤다니! 내가 미쳤다니!
    \end{dialogue}
    \begin{dialogue}{최진수}
        아직 기억 못하나본데 내가 더 중요한 사실을 알려주지.
        \paren{은밀하게}
        너 인터넷 뱅킹 할려고 너의 그 잘난 컴퓨터에 뭘 설치한 줄 아나?
    \end{dialogue}
    \begin{dialogue}[두려워하며]{김석호}
        뭐... 뭐야? 또 무슨 헛소리를 할려는거야?
    \end{dialogue}
    \begin{dialogue}{최진수}
        니가 그날 깐 것은
        \paren{다가서며 화면 클로우즈업}
        엑티브
        \paren{서스펜스와 함께}
        엑스!
    \end{dialogue}
    
    김석호는 미친 사람처럼 좌절하고 주위 직원들은 모두 걱정스러운 표정으로 지켜본다. 화면이 김석호를 가깝게 비추며 나레이션이 이어진다.
    \begin{dialogue}{김석호}
        V.O. 나... 나는 미쳐가고 있다. 이게 끝인가. 이게 나의 몰락인가.
        \paren{절망하며}
        V.O. 나는 끝났구나. 저놈이 기어코 나를 말로 죽이는구나.
    \end{dialogue}
    \intextslug[DAY]{사방이 모두 하얀 공간}
    
    멀리서 흰 옷을 걸친 흑인이 다가오는 모습이 뿌옇게 비친다.
    \begin{dialogue}[모든것을 초탈한 사람처럼 미소를 지으며]{김석호}
        아... 내가 결국 죽었구나. 그리고 여기가 천국이구나. 그리고 신은... 신은...
        \paren{놀라움과 반가움에 얼굴을 감싸며}
        신은 모건 프리먼이였구나!
    \end{dialogue}
    
    흑인이 점점 더 다가오면서 모습이 또렷해진다. 흑인임은 맞지만 모건 프리먼은 분명 아니다.
    
    \begin{dialogue}[크게 실망하며]{김석호}
        아... 아니네.
    \end{dialogue}
    
    흑인이 김석호 바로 앞 까지 다가와서 말한다.
    \begin{dialogue}{신}
        뭘 그렇게 실망을 하고 그러냐. 내가 바로 니가 두려워하던 절대자다.
    \end{dialogue}
    \begin{dialogue}[못 믿겠다는 표정으로]{김석호}
        당신이 절대자라면 어디 그 절대적인 힘을 발휘해 보시오
    \end{dialogue}
    \begin{dialogue}{신}
        쯧쯧쯧. 그렇게 나를 못 믿겠단 말이냐. 그럼 할 수 없지. 그대가 원하는 게 무엇인가?
    \end{dialogue}
    \begin{dialogue}[잠시 고민하다가]{김석호}
        나를 천국으로 보내주시오
    \end{dialogue}
    
    신은 천천히 김석호를 주시하다가 이내 웃으면서 답한다.
    
    \begin{dialogue}{신}
        천국이라? 알겠네 하하하. 그대는 평행우주론에 대해서 들어본 적 있나?
    \end{dialogue}
    
    \begin{dialogue}{김석호}
        양자역학에서 다세계 해석을 말씀하시는 거요?
    \end{dialogue}
    
    화면에 두 개의 세계가 나타난다. 두 세계 실험실에서 슈뢰딩거가 상자를 열고 있다. 한쪽에서는 상자 안에서 고양이가 살아서 나오고 슈뢰딩거가 기뻐하며 고양이를 품는다. 다른 쪽에서는 상자 안을 쳐다본 슈뢰딩거가 슬픔에 잠기며 주저앉는다. 잠시 후 두 개의 세계가 화면에서 사라진다.
    
    \begin{dialogue}{김석호}
        아... 파동함수는 붕괴하는 게 아니었구나...
    \end{dialogue}
    
    \begin{dialogue}{신}
        니가 원하던, 니가 가고싶었던 그런 대체 우주에 널 보내주겠다.
    \end{dialogue}
    
    화면에 2001에서 나오는 것과 같은 검은색의 육면체가 나타나고 김석호가 빨려 들어간다.
    
    \begin{dialogue}{김석호}
        P.O.V. "My God, its full of stars."
    \end{dialogue}
    
    신이 자비로운 웃음을 지으며 고개를 끄덕인다.
    
    \fadeout
    
    \section{Act 2}
\end{document}

