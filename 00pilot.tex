% rubber: set program xelatex

\documentclass{screenplay}
\usepackage{xltxtra,fontspec,xunicode}
\setmainfont{NanumGothic}
\setmonofont{UnGraphic}
\let\oldMakeUppercase\MakeUppercase
\renewcommand\MakeUppercase[1]{
    {\oldMakeUppercase{\underline{#1}}}
}
\title{프로그래머들 (The Programmers) -- Pilot}

\author{이인호}

\address{
chajath@gmail.com
}

\begin{document}
    \coverpage
    
    ACT 1
    
    \fadein
    \intslug[DAY]{회사 개발 사무실}
    개발 사무실에서 김석호와 최진수가 작업에 몰두하고 있다. 
    \begin{dialogue}[짜증 내는 말투로]{김석호}
        아침부터 되는 일이 없네.
    \end{dialogue}
    \begin{dialogue}{최진수}
        왜 뭐가 잘 안되?
    \end{dialogue}
    \begin{dialogue}{김석호}
        어. LaTeX 컴파일 에러가 나는데 어디가 잘못됐는지 아무리 봐도 모르겠네. 패키지가 다 설치되 있는 것 같은데 자꾸 에러가 나. 
    \end{dialogue}
    최진수 한숨을 쉬며 자리에서 일어나서 김석호 자리로 온다.
    \begin{dialogue}[어이없다는 듯이]{최진수}
        그러니까 왜 쓸데없이 LaTeX를 써? 그건 대학교 연구실에서 쓰는 거고 여긴 회사야. 이렇게 LaTeX으로 문서화를 하면 후임자들이 이걸 관리해야 되잖아. 너 개인 취향만 따져서 이런 걸 쓰면 어떻게. 
    \end{dialogue}
    \begin{dialogue}[반색하며]{김석호}
        난 프로그래머야. 회사가 정신이 돌지 않는 이상 내 후임자도 프로그래머일 것이고, 그런데 뭐가 문제야. LaTeX 못쓰는 프로그래머도 있어?
        \paren{잠시 머뭇거리다}
        근데 넌 LaTeX쓰는거 한번도 못 봤다? 너 LaTeX쓸줄 알지?
    \end{dialogue}
    \begin{dialogue}{최진수}
        야 내가 언제 못쓴다고 했냐 그냥 회사에서는 잘 안 쓰는 툴이라 이거지.
        \paren{헛기침}
        어쨌든, 근데 뭘 만들려고 LaTeX을 써?
    \end{dialogue}
    \begin{dialogue}[웃으며]{김석호}
        일기! 이제 매일 매일 일기를 쓸려고.
    \end{dialogue}
    \begin{dialogue}[황당하다는 듯이]{최진수}
        참네. 일기를 쓰는데 LaTeX이 왜 필요해?
    \end{dialogue}
    \begin{dialogue}[순진하게]{김석호}
        일기에 당연히 LaTeX을 써야지! 안 그러면 선형대수학 공식이나 항렬은 어떻게 집어넣을껀데?
    \end{dialogue}
    \begin{dialogue}{최진수}
        아니 일기장에 도대체 그런게 왜 들어가? 넌 도대체 무슨 생각을 하면서 사는 거야?
    \end{dialogue}
    \begin{dialogue}{김석호}
        \paren{잠시 최진수의 시선을 의식하며}
        그런 한심하다는 듯한 얼굴 표정 짓지 마. 내가 일기장에 뭘 쓰던지, 그건 내 자유잖아? 그런 걸로 사람을 판단하지 마!
    \end{dialogue}
    \begin{dialogue}{최진수}
        너무 민감하게 생각하지는 마. 하긴 너처럼 현실과 동떨어진 세계관을 가지고 살아가려면 그런 것도 필요한지도 모르겠네.
    \end{dialogue}
    \begin{dialogue}{김석호}
        아니 내가 현실과 동떨어진 세계관을 가졌다니 무슨 말을 하는 거야?
    \end{dialogue}
    두 사람이 대화를 하는 사이 주위 직원들이 하나 둘 모여든다.
    \begin{dialogue}{최진수}
        세상은 니가 생각하는 그런 곳이 아니야. 지금 제일 많이 쓰이는 운영체제가 뭐야?
    \end{dialogue}
    \begin{dialogue}{김석호}
        우분투!
    \end{dialogue}
    \begin{dialogue}{최진수}
        땡! 윈도우즈!
        \paren{김석호 좌절하는 표정}
        제일 많이 쓰는 웹 브라우저는?
    \end{dialogue}
    \begin{dialogue}{김석호}
        파이어폭스!
    \end{dialogue}
    \begin{dialogue}{최진수}
        틀렸어! 인터넷 익스플로러야! 
        \paren{사악한 미소를 지으며}
        너 혹시 인터넷 뱅킹 해봤어?
    \end{dialogue}
    주위에 모인 직원들이 최진수를 말린다
    \begin{dialogue}{직원1}
        그만해 최진수. 왜 일을 키우는 거야?
    \end{dialogue}
    \begin{dialogue}{직원2}
        동료한테 왜 이렇게 잔인하게 구는 거야?
    \end{dialogue}
    \begin{dialogue}{김석호}
        당연히 해봤지. 그건 왜?
    \end{dialogue}
    \begin{dialogue}{최진수}
        그때 넌 무슨 브라우저를 쓰고 있었지?
    \end{dialogue}
    \begin{dialogue}{김석호}
        음... 브라우저? 뭘 쓰고 있었더라.
    \end{dialogue}
    \begin{dialogue}{최진수}
        당연히 기억 못하겠지. 왜 그런지 알아?
        \paren{김석호 고개를 젓는다}
        그건 그 기억이 너무 트라우마틱해서 무의식적으로 기억을 짓누르고 있기 때문이야. 
        \paren{손가락으로 머리를 가리키며}
        너의 방어기제가 작동하고 있는거지.
    \end{dialogue}
    \begin{dialogue}[서서히 고통에 잠기면서 얼굴을 손으로 감싼다]{김석호}
        무슨.. 무슨 소리를 하고있는 거야 지금?
    \end{dialogue}
    \begin{dialogue}{최진수}
        너는 인터넷 뱅킹을 할려고 인터넷 익스플로러를 썼었어.
    \end{dialogue}
    \begin{dialogue}{김석호}
        그럴리가... 그럴리가 없어. 네 이놈 그 입 다물지 못할까!
    \end{dialogue}
    \begin{dialogue}{최진수}
        이거 봐 말투부터 이상해지고 있잖아. 그게 왜 그런지 알아?
        \paren{다가가서 속삭이는 말투로}
        그건 니가 미쳐가고 있기 때문이야
    \end{dialogue}
    주위에 직원들 웅성거리는 소리. 몇몇 직원은 최진수의 팔을 붙들며 말리는 시늉을 한다.
    \begin{dialogue}{김석호}
        하하하하. 내가 미쳤다고. 내가 미쳤다고!
        \paren{팔을 휘두르며}
        미친 건 내가 아니라 너야! 내가 미쳤다니! 내가 미쳤다니!
    \end{dialogue}
    \begin{dialogue}{최진수}
        아직 기억 못하나본데 내가 더 중요한 사실을 알려주지.
        \paren{은밀하게}
        너 인터넷 뱅킹 할려고 너의 그 잘난 컴퓨터에 뭘 설치한 줄 아나?
    \end{dialogue}
    \begin{dialogue}[두려워하며]{김석호}
        뭐... 뭐야? 또 무슨 헛소리를 할려는거야?
    \end{dialogue}
    \begin{dialogue}{최진수}
        니가 그날 깐 것은
        \paren{다가서며 화면 클로우즈업}
        엑티브
        \paren{서스펜스와 함께}
        엑스!
    \end{dialogue}
    
    김석호는 미친 사람처럼 좌절하고 주위 직원들은 모두 걱정스러운 표정으로 지켜본다. 화면이 김석호를 가깝게 비추며 나레이션이 이어진다.
    \begin{dialogue}{김석호}
        V.O. 나... 나는 미쳐가고 있다. 이게 끝인가. 이게 나의 몰락인가.
        \paren{절망하며}
        V.O. 나는 끝났구나. 저놈이 기어코 나를 말로 죽이는구나.
    \end{dialogue}
    \extslug[DAY]{사방이 모두 하얀 공간}
    
    멀리서 흰 옷을 걸친 흑인이 다가오는 모습이 뿌옇게 비친다.
    \begin{dialogue}[모든것을 초탈한 사람처럼 미소를 지으며]{김석호}
        아... 내가 결국 죽었구나. 그리고 여기가 천국이구나. 그리고 신은... 신은...
        \paren{놀라움과 반가움에 얼굴을 감싸며}
        신은 모건 프리먼이였구나!
    \end{dialogue}
    
    흑인이 점점 더 다가오면서 모습이 또렷해진다. 흑인임은 맞지만 모건 프리먼은 분명 아니다.
    
    \begin{dialogue}[크게 실망하며]{김석호}
        아... 아니네.
    \end{dialogue}
    
    흑인이 김석호 바로 앞 까지 다가와서 말한다.
    \begin{dialogue}{신}
        뭘 그렇게 실망을 하고 그러냐. 내가 바로 니가 두려워하던 절대자다.
    \end{dialogue}
    \begin{dialogue}[못 믿겠다는 표정으로]{김석호}
        당신이 절대자라면 어디 그 절대적인 힘을 발휘해 보시오
    \end{dialogue}
    \begin{dialogue}{신}
        쯧쯧쯧. 그렇게 나를 못 믿겠단 말이냐. 그럼 할 수 없지. 그대가 원하는 게 무엇인가?
    \end{dialogue}
    \begin{dialogue}[잠시 고민하다가]{김석호}
        나를 천국으로 보내주시오
    \end{dialogue}
    
    신은 천천히 김석호를 주시하다가 이내 웃으면서 답한다.
    
    \begin{dialogue}{신}
        천국이라? 알겠네 하하하. 그대는 평행우주론에 대해서 들어본 적 있나?
    \end{dialogue}
    
    \begin{dialogue}{김석호}
        양자역학에서 다세계 해석을 말씀하시는 거요?
    \end{dialogue}
    
    화면에 
    
    \slug{두 개의 세계}{}{}
    
    가 나타난다. 두 세계 실험실에서 슈뢰딩거가 상자를 열고 있다. 한쪽에서는 상자 안에서 고양이가 살아서 나오고 슈뢰딩거가 기뻐하며 고양이를 품는다. 다른 쪽에서는 상자 안을 쳐다본 슈뢰딩거가 슬픔에 잠기며 주저앉는다. 잠시 후 두 개의 세계가 화면에서 사라진다.
    
    \slug{다시 흰 공간}{}{}
    
    \begin{dialogue}{김석호}
        아... 파동함수는 붕괴하는 게 아니었구나...
    \end{dialogue}
    
    \begin{dialogue}{신}
        니가 원하던, 니가 가고싶었던 그런 대체 우주에 널 보내주겠다.
    \end{dialogue}
    
    화면에 2001에서 나오는 것과 같은 검은색의 육면체가 나타나고 김석호가 빨려 들어간다.
    
    \begin{dialogue}{김석호}
        P.O.V. "My God, it's full of stars."
    \end{dialogue}
    
    신이 자비로운 웃음을 지으며 고개를 끄덕인다.
    
    \fadeout
    
    \pagebreak
    
    ACT 2
    
    \fadein
    
    \extslug[DAY]{GNU 판타지 랜드}
    
    GNU 판타지 랜드의 전경이 펼쳐진다. 잔디 동산 위로 푸른 하늘이 펼쳐진다. 평화로운 전경이 다소 동화적이다. 리쳐드 스톨먼이 노트북으로 작업에 몰두하고 있다. 김석호가 판타지 랜드에 갑작스레 나타난다. 리쳐드 스톨먼이 김석호 쪽으로 다가온다.
    
    \begin{dialogue}{리처드 스톨먼}
        자네는 누구인가?
    \end{dialogue}
    
    \begin{dialogue}{김석호}
        아니, 리처드 스톨먼 아니오? 저는... 프로그래머입니다.
    \end{dialogue}
    
    \begin{dialogue}{리처드 스톨먼}
        오오, 프로그래머셨오? 이리 오시오. 자 여기.
    \end{dialogue}
    
    리처드 스톨먼은 가지고 있던 노트북을 김석호에게 건네고 옆에서 리코더를 집어 들고 연주를 하기 시작한다.
    
    \begin{dialogue}[한참 이것 저것 살펴보다가]{김석호}
        이 우주에서는, 우분투 리눅스가 대세인가요?
    \end{dialogue}
    \begin{dialogue}[리코더 연추를 멈추고]{리처드 스톨먼}
        우분투? 리눅스?
        \paren{잠시 생각에 잠시다가}
        아, 리눅스라고 잠시 유행한 커널이 있었지. 지금은 전부 GNU Hurd 커널을 쓰네. Hurd 라고 들어보았나?
    \end{dialogue}
    \begin{dialogue}[감격해 하며]{김석호}
        GNU Hurd... 그게 정녕 쓰인단 말입니까?
    \end{dialogue}
    \begin{dialogue}[당연하다는 표정으로]{리처드 스톨먼}
        그럼. 내 전화기에서도 돌아가는 걸?
    \end{dialogue}
    
    리처드 스톨먼이 전화기를 꺼내서 보여준다. 전화기 화면에 Bash 같은 CLI 쉘이 나온다.
    
    \begin{dialogue}{리처드 스톨먼}
        흠... 잠시만 어디 문자 보낼께 있어서
    \end{dialogue}
    
    \slug{스톨먼의 전화기}{}{}
    
    리처드 스톨먼이 전화기를 조작하자 화면에 emacs 가 뜬다.
    
    \begin{dialogue}{김석호}
        V.O. 스마트폰에 Hurd 가 돌아가고 문자를 보내는데 emacs 를 쓰다니. 아, 여기가... 여기가 천국이구나. 오... 신이시여...
    \end{dialogue}
    
    \slug{다시 GNU판타지랜드}{}{}
    
    리처드 스톨먼은 한참 전화기를 조작하고 나서 전화기를 김석호에게 건낸다.
    
    \begin{dialogue}{리처드 스톨먼}
        자, 한번 구경해보시게
    \end{dialogue}
    \begin{dialogue}{김석호}
        내, 감사합니다. 어디보자, 어플은 dpkg 로 다 깔고, 웬만한 조작한 쉘 스크립트로 다 하고...
    \end{dialogue}
    
    한참 호기심 어린 눈으로 조작하다가 화면에 emacs 가 뜨자 머뭇거린다.
    
    \begin{dialogue}[그 모습을 빤히 보고 있다가]{리처드 스톨먼}
        프로그래머라고 하더니 emacs 를 잘 못쓰나보네?
    \end{dialogue}
    \begin{dialogue}{김석호}
        아니 그게 아니라, 써본지가 좀 오래되서요.
        \paren{작은 목소리로}
        사실 써본적도 별로 없고
    \end{dialogue}
    \begin{dialogue}{리처드 스톨먼}
        아니 그렇다면 문서편집할때 뭘 쓴단 말인가?
    \end{dialogue}
    \begin{dialogue}[난처해하며]{김석호}
        전... 사실... 학생때부터 쭉... vi 를 써왔습니다.
    \end{dialogue}
    \begin{dialogue}[놀라며]{리처드 스톨먼}
        vi 라니... vi 라니...
        \paren{노여워하며}
        VI 라니!
    \end{dialogue}
    
    GNU 판타지랜드 하늘에 먹구름이 끼면서 점차 어두워진다. 천둥번개가 친다.
    
    \begin{dialogue}{리처드 스톨먼}
        자네는 프로그래머라고 하지 않았나? 어떻게 이럴 수 있지? 가만... 지금 입고 있는 옷!
    \end{dialogue}
    \begin{dialogue}{김석호}
        네... 네...?
    \end{dialogue}
    \begin{dialogue}{리처드 스톨먼}
        그 옷에 박혀있는 그 그림?
        \paren{심각한 표정을 지으며}
        그거, 오픈소스인가?
    \end{dialogue}
    \begin{dialogue}{김석호}
        아니요... 그냥 시장에서 산거니까
        \paren{기어들어가는 목소리로}
        누군가의 지적 재산이겠죠
    \end{dialogue}
    \begin{dialogue}{리처드 스톨먼}
        뭐라 지적 재산? 그럼 그림을 지우고 오픈소스로 다시 그렸어야지! 지적 재산을 입고 다니다니! 허허!
        지금 차고 있는 시계, 그 설계도는 오픈소스인가?
    \end{dialogue}
    \begin{dialogue}{김석호}
        아니요 이것도 지적 재산입니다.
    \end{dialogue}
    \begin{dialogue}{리처드 스톨먼}
        그럼 지금 신고있는 신발은? 그것도 오픈소스가 아닌가?
    \end{dialogue}
    \begin{dialogue}{김석호}
        ... 네...
    \end{dialogue}
    \begin{dialogue}{리처드 스톨먼}
        자네 어떻게 그렇게 당당하게 말할 수 있는가? 자네 직업이 혹시 프로그래머가 아니라... 상용 프로그래머인가?
    \end{dialogue}
    \begin{dialogue}[낙담한듯]{김석호}
        그래요. 저는 저의 저작물을 재산으로 삼고 유저들을 노예로 삼는 무지막지한 상용 프로그래머입니다
    \end{dialogue}
    \begin{dialogue}{리처드 스톨먼}
        이런! 이럴 수가... 이럴 수가...
        \paren{슬픈 표정을 짓는다}
        자네가 이런 사람이였다니!
    \end{dialogue}
    하늘은 완전히 어두워지고 천둥소리는 더 커져 간다. 바닥에 잔디는 모두 말라 죽어버리고 용암이 땅을 침식한다. 리처드 스톨먼은 어느새 제다이 도복을 입고 있다.
    \begin{dialogue}[소리치며]{리처드 스톨먼}
        넌 선택받은 자였어!
    \end{dialogue}
    장중한 음악이 흐르고 김석호는 바닥에 엎드려 흐느낀다.
    \begin{dialogue}{김석호}
        V.O. 리처드 스톨먼이 제다이였구나.
    \end{dialogue}
    \fadeout
    
    \pagebreak
    
    ACT 3
    
    \fadein
    \intslug[DAY]{회사 개발 사무실}
    
    사무실은 어수선한 분위기이다.
    
    \slug{한쪽에서는}{}{} 
    
    김석호가 정신을 잃고 쓰러져있고 최진수와 주위 동료들이 걱정스러워 하는 반면에 
    
    \slug{사무실의 다른 모퉁이에서는}{}{}
    
    장사장과 정팀장이 이야기를 나누고 있는 가운데 신종훈이 즐거운 표정으로 등장한다.
    
    \begin{dialogue}{장사장}
        아니 신종훈이 뭐가 그렇게 즐거워?
    \end{dialogue}
    \begin{dialogue}[손에 쥔 USB를 들어보이며]{신종훈}
        인도에 아웃소싱 거래처에서 새 버전을 보내왔어요
    \end{dialogue}
    \begin{dialogue}[반가워하며]{정팀장}
        아아! 드디어 버젼 13.0이 나왔단 말인가! 이젠 정말로 고생 끝 행복 시작이구나!  
    \end{dialogue}
    \begin{dialogue}[감격에 겨워 눈물을 글썽이며]{장사장}
        6년의 개발 기간. 투입된 자본금만 수십 억원. 그간 6개월마다 퇴사하던 직원들의 얼굴이 한 명 한 명 떠오르는 듯 하구나.\\
        소프트웨어! 그대는 망망대해를 휘젓고 다니는 흰 고래 이고 나는 Ahab 선장이다.
        \paren{신종훈이 들고있던 USB를 가져다 들며}
        이 USB는 내 심장에서 뿜어져 나오는 포경 작살이다! 어서... 어서 설치를 해봐! 어서!
    \end{dialogue}
    장사장이 USB를 꼽고 정팀장이 쳐다보는 가운데 신종훈이 열심히 키보드를 친다.
    \begin{dialogue}{신종훈}
        이제 컴파일 해보겠습니다!
    \end{dialogue}
    신종훈이 키보드 조작을 마치자 정팀장이 둘에게 눈치를 준다. 셋은 눈을 지긋이 감고 두 손을 공손하게 모은다.
    \begin{dialogue}[기도하듯]{정팀장}
        Java신이시여 우리 벤처기업을 어여삐 여기소서...
    \end{dialogue}
    
    정팀장이 진지한 표정으로 계속 기도를 하고 배경에는 성스러운 음악이 깔린다. 잠시 후 기도가 끝난다.
    
    \begin{dialogue}{장사장}
        어떻게 됐나?
    \end{dialogue}
    \begin{dialogue}[한참 화면을 주시하다가]{신종훈}
        컴파일... 오류 입니다.
    \end{dialogue}
    \begin{dialogue}{정팀장}
        컴파일 오류라니... 어떻게 버젼 13.0 이 될 동안 컴파일이 안된단 말인가?
    \end{dialogue}
    \begin{dialogue}{장사장}
        아아... 이렇게... 우리는 이렇게 망하는 건가... 이제 돈 긁어모을 닷컴버블도, 정부 지원금도, 벤처 투자자도 없네... 이를 어쩌지 이를 어쩌지 허허...
    \end{dialogue}
    
    장사장은 뒷짐을 지고 천천히 걸어서 사장실에 들어간다. 사장실 문이 닫히자 마자
    
    \begin{dialogue}[괴성을 지른다]{장사장}
        O.S. 으악! 으악! 으악! 다 고쳐내! 고쳐내!!
    \end{dialogue}
    
    다들 난처한 표정을 짓는다.
    
    \begin{dialogue}{정팀장}
        이 난국을 타개하기 위해선 그 분을 모셔오는 수 밖에 없다.
    \end{dialogue}
    \begin{dialogue}{신종훈}
        그 분이 누구입니까?
    \end{dialogue}
    \begin{dialogue}[한심하다는 듯이]{정팀장}
        입사한지 얼마가 지났는데 아직 그 분이 누군지도 모른단 말인가?
    \end{dialogue}
    \begin{dialogue}{신종훈}
        저 두달 됐는데요?
    \end{dialogue}
    \begin{dialogue}[한숨쉬며]{정팀장}
        아... 그랬나?
        \paren{잠시 생각하다}
        예전부터 업계에서 전설로 통하는 분이 계시다.
        \paren{동네 치킨집 전단을 주며}
        여기 후라이드 치킨 한마리 시키시게
    \end{dialogue}
    \begin{dialogue}[의아해하며]{신종훈}
        예? 지금 치킨을 시켜요?
    \end{dialogue}
    \begin{dialogue}[귀찮다는 말투]{정팀장}
        그냥 하라면 해! 가뜩이나 컴파일도 안되는데...
    \end{dialogue}
    \begin{dialogue}{신종훈}
        네..네.. 알았어요
        \paren{책상위 전화를 집어들고 닭을 주문한다}
    \end{dialogue}
    \fadeout
    
\end{document}

